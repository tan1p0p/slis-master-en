\documentclass[a4paper]{report}

\usepackage{masterThesisJa-En}
\setcounter{tocdepth}{3}
\setcounter{page}{-1}

% 【必須】主題:\maintatile{日本語}{英語}
\maintitle{国文学論文の情報検索}{Information Retrieval for Articles on Japanese Literature}

% 【任意】副題:\subtitle{日本語}{英語}
% 副題が不要な場合は次の行をコメントアウトしてください
\subtitle{—図書館情報メディア研究科の場合—}{: A Case Study at the Graduate School of Library, Information, and Media Studies}

% 【必須】発表年月:\publish{年}{月}
% 月は英語表記(Marchなど)にしてください
\publish{Year}{Month}

% 【必須】学生情報:\student{学籍番号}{氏名(英語:Twins登録の表記)}
\student{20XX21XXX}{Tsukuba Taro}

% 【必須】概要:\abst{概要}
\abst{
In this research, we developed a mobile phone using a touch panel. A traditional cellular phone has a numerical keypad, a cross key, and several buttons, and users operated cellular phones by using them. Our new cellular phone has only three buttons, and users operate our new phone through the touch panel. We report that the new mobile phone is easier to use than traditional mobile phones.
}

% 【必須】研究指導教員:\advisors{主研究指導教員}{副研究指導教員}
\advisors{Ichiro DAIGAKU}{Hanako SHIHO}


% 以下,本文を出力
\begin{document}

\makecover

\addtolength{\textheight}{-5mm}	% 本文の下限を5mm上昇
\setlength{\footskip}{15mm}	% フッタの高さを15mmに設定
\fontsize{11pt}{15pt}\selectfont

% 目次・表目次を出力
\pagebreak\setcounter{page}{1}
\pagenumbering{roman} % I, II, III, IV 
\pagestyle{plain}
\tableofcontents
\listoffigures

% 本文
\parindent=1zw	% インデントを1文字分に設定
\pagebreak\setcounter{page}{1}
\pagenumbering{arabic} % 1,2,3
\pagestyle{plain}

% 章:\chapter{}
% 節:\section{}
% 項:\subsection{}

\chapter{Introduction}
% プレースホルダ
Currently, we can communicate easily through mobile phones. Users can talk anywhere using mobile phones. Therefore, it is easier for users to interact with business partners and friends using a mobile phone than using a fixed or public phone. \cite{xu2017deep}

\chapter{Related Work}
\section{Study of Aa}
\subsection{Study of Bb}

\chapter{Proposal}

\chapter{User Study}

\chapter{Disucussion}

\chapter{Conclusion}

\chapter*{Acknowledgement}
\addcontentsline{toc}{chapter}{\numberline{}Acknowledgement}


% 参考文献(References)
\newpage
\addcontentsline{toc}{chapter}{\numberline{}References}
\renewcommand{\bibname}{References}

%% 参考文献に bibtex を使う場合
\bibliographystyle{junsrt}
\bibliography{ref}

%% 参考文献を直接ファイルに含めて書く場合
% \begin{thebibliography}{99}

% % e.g.)
% \bibitem{studyA}
% Hammerson, Geoffrey A., Connecticut wildlife : biodiversity, natural history, and conservation, Hanover: University Press of New England, 2004, p.89. 

% \end{thebibliography}

\end{document}